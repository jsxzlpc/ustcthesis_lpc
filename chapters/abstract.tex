%!TEX root =  ../main.tex

\begin{abstract}
  企业营销是企业的一项整体性、长期性的经营活动,贯穿于企业经营的全过程,营销策略的好坏严重影响着企业的生存和发展。特别是在当下如此激烈的市场竞争中,一种高效的营销决策方式对企业快速开拓市场、提高整体效益、增强自身竞争力具有重要作用。所以,近年来,随着信息化和智能化的快速发展,越来越多的企业希望借助人工智能的力量进行营销决策。特别地,在营销决策时,企业追求的最终目标是长期、整体收益最大化,而不仅仅局限在即时收益上。

  强化学习是机器学习的重要组成部分,主要用于解决序贯决策问题。它通过智能体不断地与环境进行交互,并从环境反馈的延迟奖赏中学习状态与行为之间的映射关系,以使的累积奖赏最大化。近年来,随着AlphaGo的成功,强化学习技术得到了学术界和工业界的广泛的关注,并且取得了令人振奋的表现。但是,大部分的研究和应用集中于机器人、游戏等领域,针对其它领域的研究和应用则相对较少。考虑到企业营销的过程也是一个序列决策过程,并且其追求的长期收益最大化与强化学习累积奖赏最大化的目标不谋而合,因此,用强化学习技术解决企业营销决策问题中有着天然的优势。

  关系市场营销主要是指企业与目标顾客之间的关系以及企业与营销渠道之间的关系等。前者的应用场景如“直效营销”,后者的应用场景如“广告渠道预算分配”。本文从基于值函数逼近的强化学习模型出发,着眼于模型在以上两个场景的应用中存在的问题,提出相应的改进方法,最后,通过仿真实验验证了模型的有效性。本文的主要研究内容包括以下三个方面:

  (1)在直效营销场景中,因为序列之间的存在影响,导致传统的深度强化学习不能较好的进行值函数的逼近。因此,本文提出了基于RNN的深度强化学习混合模型。该混合模型由两个网络构成,其中,利用RNN网络的长时依赖性可以从监督数据中较好的学习到观测信息的隐状态,然后将其作为状态输入,再利用DQN网络强大的逼近能力进行强化学习。两个网络协同控制与优化,可以在不需要借助专家领域知识的情况下,提高值函数的学习精度,以更好的实现直效营销场景中客户生命价值最大化的目标。与其它深度强化学习相比,该模型没有直接将环境的观测值作为强化学习的状态输入,而是将观测值的隐状态作为输入,并且为了进一步提高学习精度,通过改变网络结构,提出了一步混合模型和两步混合模型。

  (2)在广告渠道预算分配场景中,因为存在数据量较少以及渠道间的时空影响等问题,导致传统的值函数逼近方法不能较好的进行值函数的学习。因此,本文提出了SVR-Q-MCKP分块逼近强化学习模型。其中,利用基于径向基核函数的SVR方法将强化学习中的值函数逼近转化为高维空间中的线性回归问题,以提高模型的学习速度;为了提高模型的逼近精度,提出了分块学习的强化学习思想。另外,通过考虑渠道之间的时空营销、探索与利用之间的平衡关系等问题,提出改进的值函数的更新方法,最后,将带约束的预算分配问题与多选择背包问题相结合,得出最后的分配策略。

  (3)实验验证阶段。将基于RNN的深度强化学习混合模型以及其它基准模型应用在直邮营销场景中,根据KDD-CUP-1998数据集进行模型训练,并且构建仿真环境,比较、验证模型的效果。将SVR-Q-MCKP分块逼近强化学习模型以及其它基准模型应用在基于DSP平台上的真实投放数据训练模型,并基于此数据构建仿真环境,最后在此仿真环境中比较、验证模型的效果。

  \keywords{强化学习;深度学习;函数逼近;企业营销;直效营销;渠道预算分配}
\end{abstract}

\begin{enabstract}
  This is a sample document of USTC thesis \LaTeX{} template for bachelor,
  master and doctor. The template is created by zepinglee and seisman, which
  orignate from the template created by ywg. The template meets the
  equirements of USTC theiss writing standards.

  This document will show the usage of basic commands provided by \LaTeX{} and
  some features provided by the template. For more information, please refer to
  the template document ustcthesis.pdf.

  \enkeywords{University of Science and Technology of China (USTC); Thesis;
  \LaTeX{} Template; Bachelor; Master; PhD}
\end{enabstract}
