\chapter{绪论}
 % 简短的介绍

\section{研究背景及意义}
 % 1、CRM出现的背景、含义和意义(适当的拔高)、进而引出ltv最大化和广告投放
 % 2、强化学习的出现的背景以及为什么可以用于解决CRM问题,但是解决这个问题还有什么问题。
在中国改革开放和国际化进程的不断深入下,我们对市场、对营销、对管理的态度发生了从无知到有知,从漠视到重视的巨大转变。时至今日,我们发现我们好像从来没有与世界的脉搏如此的接近过,同样,我们的企业和国家也从来没有如此深切的感受到全球化的竞争是如此残酷。因为市场竞争的不断加剧,每个企业都在努力寻找着自己的核心竞争能力,以在竞争中取得优势,使企业不断的发展壮大。但是,近年来,信息技术的广泛使用,使得众多行业的产品在价格、质量和服务上的差异越来越小,如何在如此激烈的市场竞争中领先对手,给当前的商业研究提出了新的要求,客户关系管理(Customer Relationship Management, CRM)就是其中一个重要的研究方向。

CRM一词最初由Gartner Group\footnote{高德纳咨询公司,全球第一家信息技术研究和分析的公司,现已发展成一家独立的咨询公司}在1999年提出,但是因为商业场景的复杂性和广泛性,不同的学者和商业机构从不同角度给出了不同的定义。结合Gartner Group和CRMguru.com\footnote{http://crmguru.com/:提供CRM相关文章、业界信息以及专业人员讨论区的网站}等公司对CRM的解释,一种可接受的定义为:CRM是现代信息技术、经营理念和管理思想的结合体,它以信息技术为手段,对现实的和潜在的客户关系以及业务伙伴关系进行多渠道管理,形式一个自动化的解决方案,最终实现业务操作效益的提高和利润的增长\citep{客户关系管理丁建石}。
其中,Customer代表的意义较为广泛,它包括了现实已存在的和潜在的客户以及业务伙伴。

CRM的所涉及的研究内容非常广泛,研究方向也朝着多元化的趋势发展\citep{王广宇2013客户关系管理}。在过去的二十年里,随着互联网和数据库的广泛应用,各个行业、各个公司每天都会积累一定的商业数据,如何将这些原始数据进行实时的和更深层次的分析,以辅助或代替企业进行商业决策,成为近年来CRM领域最为活跃的研究内容。其中,最有代表性的方向是基于机器学习技术与CRM的融合,利用机器学习技术可以从数据中自动提取出其背后隐含信息,然后更好的利用这些信息进行预测、分析和决策,使得CRM的全部功能得以有效发挥。

强化学习(Reinforcement Learning, RL)技术是机器学习的重要组成部分,主要用于解决序贯决策问题。它通过智能体Agent不断地与环境进行交互,并从环境反馈的延迟回报中学习状态与行为之间的映射关系,以使得回报达到长期最大化,即通过不断的学习,可以输出使得长期收益最大化的策略,这与CRM所追求的“产品”生命周期价值最大化的目标不谋而合。强化学习目前主要应用于机器人、游戏等领域,并都取得了令人振奋的表现。但是,因为其在真实的应用中容易出现不收敛或收敛速度慢、环境状态部分可观察、仿真环境很难建立等问题,针对其它领域的研究和应用则相对甚少。这也是本文采用强化学习的出发点。

本文主要关注CRM领域的两个应用场景:1)在产品的直接营销活动(Direct Marketing Campaigns)中,根据顾客信息和消费记录,如何最大化顾客的生命周期价值的问题。2)在多渠道广告投放中,根据每个渠道投放和回报数据,如何做出科学的投放策略以最大化多渠道的长期利润,因为在广告投放中一般无法接触顾客,因此属于间接营销活动。如果单单依靠市场人员进行分析判断,其工作量是巨大的而且往往产生不了较好的决策或者很难持续产生较好的决策。所以,通过强化学习技术,可以在自动化产生科学决策的同时使得企业的收益达到长期最大化,就变的十分重要和关键了。

通过研究直接营销活动中顾客生命周期价值最大化的问题,建立企业同顾客之间的长期关系,帮助企业分析每一位顾客在每一阶段的特点,对是否应该向该顾客提供营销信息或者应该提供什么样的营销信息做出科学判断,进而可以创造出不断增加的忠诚客户和更大的盈利空间。通过研究多渠道广告投放的应用,可以帮助企业了解每一个投放渠道在每一个阶段对企业的价值,进而可以做出智能化的投放策略,最大化每个渠道的潜力。总之,研究CRM的相关应用,可以提升客户关系的管理水平、树立企业新型营销理念、降低企业成本、提高企业效益,进而全面提高企业的核心竞争力,因此针对CRM领域的研究具有重要的研究前景和社会价值,同时,针对强化学习的研究和改进并将拓宽其应用领域,对强化学习的发展也有着重要而深远的意义。

\section{国内外研究现状}

\subsection{CRM的研究现状}
% 介绍crm目前的研究方向,引出其与机器结合的方法,进而介绍机器学习技术在定向促销和多渠道广告投放中的应用
虽然CRM一词最早出现在1999年,但是相关理论的研究已有百余年之久。总的来看大概经历了CRM理念萌芽时期、CRM应用雏形时期、CRM实用化时期以及CRM智能集成时期\citep{王广宇2013客户关系管理},其中,CRM智能集成就是指将人工智能等手段集成到CRM应用中,多年来,经过众多学者和专家的不断努力,已经在此方面取得了瞩目的成果,并将其应用到各种商用CRM系统中,比如国外微软公司的Dynamics CRM、甲骨文公司的Siebel、People Soft以及惠普公司的Front Office等CRM产品,国内主要有用友公司的TruboCRM产品,立有信公司的MyCRM产品等。由于本文主要关注直接营销活动和多渠道广告投放两个应用场景,因此主要将这两个方面的研究现状进行详细介绍。
\paragraph{直接营销活动}
目前,针对产品营销推广领域的研究,主要是用监督和非监督的机器学习方法。

\paragraph{多渠道广告投放}


\subsection{强化学习研究现状}

\section{主要研究内容}

\section{本文结构安排}