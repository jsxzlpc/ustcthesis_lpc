\begin{abstract}

 直复营销是企业的一项整体性、长期性的经营活动,贯穿于企业经营的全过程,营销策略的好坏严重影响着企业的生存和发展。在营销过程时,企业所追求的目标是长期收益最大化,而传统的经典分类算法和基于代价敏感的分类算法在处理该问题时只能考虑单个决策收益最大化,因此有很大的局限性。

 % 特别是在当下如此激烈的市场竞争中,一种高效的营销决策方式对企业快速开拓市场、提高整体效益、增强自身竞争力具有重要作用。所以,近年来,随着信息化和智能化的快速发展,越来越多的企业希望借助人工智能的力量进行营销决策。特别地,

 强化学习是机器学习的重要组成部分,主要用于解决序贯决策问题。它通过智能体不断地与环境进行交互,并从环境反馈的延迟奖赏中学习状态与行为之间的映射关系,以使的累积奖赏最大化。考虑到企业营销的过程也是一个序贯决策过程,并且其追求的长期收益最大化与强化学习累积奖赏最大化的目标不谋而合,因此,使用强化学习技术解决企业营销决策问题中有着天然的优势。为了更好的适应实际需求,本文针对直复营销场景中的营销决策点不固定、数据负载大学习效率不高以及客户状态的部分可观测性等三个方面对基于值函数的强化学习算法进行改进,并使用仿真环境验证所提方法的有效性。具体如下:

 % 近年来,随着AlphaGo的成功,强化学习技术得到了学术界和工业界的广泛的关注,并且取得了令人振奋的表现。但是,大部分的研究和应用集中于机器人、游戏等领域,针对其它领域的研究和应用则相对较少。

 % (1)在直效营销场景中,因为序列之间的存在影响,导致传统的深度强化学习不能较好的进行值函数的逼近。因此,本文提出了基于RNN的深度强化学习混合模型。该混合模型由两个网络构成,其中,利用RNN网络的长时依赖性可以从监督数据中较好的学习到观测信息的隐状态,然后将其作为状态输入,再利用DQN网络强大的逼近能力进行强化学习。两个网络协同控制与优化,可以在不需要借助专家领域知识的情况下,提高值函数的学习精度,以更好的实现直效营销场景中客户生命价值最大化的目标。与其它深度强化学习相比,该模型没有直接将环境的观测值作为强化学习的状态输入,而是将观测值的隐状态作为输入,并且为了进一步提高学习精度,通过改变网络结构,提出了一步混合模型和两步混合模型。

一方面,基于经典的强化学习算法Q-learning进行研究,结合直复营销场景中决策点间时间不固定问题以及数据规模大学习效率不高的问题,提出了改进的Q-learning算法。具体地,首先,为了减少不同决策点之间的可变时间间隔给强化学习中的奖赏信号带来的噪声影响,提出使用均值归一化的方法进行解决。接着,针对Q值函数在迭代更新过程中因为时间间隔更新不同步问题而带来的偏差影响,提出一个标准化因子,并仿照值函数更新方法进行标准化因子的更新,从而可以有效的解决以上偏差问题。最后,为了解决在直复营销场景中,随着数据量的提升,Q-learning算法更新速度慢,学习效率不高的问题,在Q-采样的基础上,引入TD偏差,提出基于TD偏差的Q-采样方法,以减少训练次数的同时提高学习效果。通过仿真实验证明,本文所提的基于可变时间间隔的Q-learning算法在不定期营销中可以取得高的收益,本文所提的基于TD偏差的Q-采样法,可以在减少采样样本数量的同时,可以获得更高的收益。

另一方面,基于深度强化学习DQN模型进行研究,针对传统强化学习Q-learning在处理直复营销场景中的客户状态的部分可观测问题时,需要引入大量专家领域知识的问题,提出了基于RNN的深度强化学习混合模型。具体地,首先,结合直复营销场景的时序特点,提出使用基于RNN的DQN模型(DQN_RNN)来解决直复营销的决策问题。然后,指出DQN_RNN模型在网络优化过程中不能很好的同时处理隐状态的学习和值函数的逼近,并由此提出基于两个网络的混合模型:通过RNN网络从监督数据中学习到隐状态的表示方法后,再将隐状态作为DQN网络的状态输入进行强化学习,通过这种方式可以使的两个网络优势互补,在达到值函数逼近效果的同时也跟好地学习到了隐状态的表示方法,从而摆脱了需要借助专家领域知识的困扰。最后,为了达到更好的策略学习目的,又根据网络结构和参数训练方式的不同提出了三个改进模型:双网络独立训练模型、一步联合训练混合模型和两步联合训练混合模型。通过仿真实验证明,本文所提的混合训练模型的方法在定期营销中可以取得更高的收益,并且不需要利用大批量数据进行训练同样可以产生较好的营销策略。

\keywords{强化学习;值函数;Q-learning;DQN;深度学习;直复营销}
\end{abstract}

\begin{enabstract}
  This is a sample document of USTC thesis \LaTeX{} template for bachelor,
  master and doctor. The template is created by zepinglee and seisman, which
  orignate from the template created by ywg. The template meets the
  equirements of USTC theiss writing standards.

  This document will show the usage of basic commands provided by \LaTeX{} and
  some features provided by the template. For more information, please refer to
  the template document ustcthesis.pdf.

  \enkeywords{University of Science and Technology of China (USTC); Thesis;
  \LaTeX{} Template; Bachelor; Master; PhD}
\end{enabstract}