\begin{acknowledgements}
三年前,怀着梦想和激情,我来到了合肥,来到了梦寐以求的中国科大,开始了研究生的学习和生活。三年间,我一直在努力,一直在拼搏,在追求梦想的道路上,我从未停歇。三年后,当我要离开校园的时候,脑海中不禁浮现出许多画面,不管是成功的喜悦还是失败的沮丧,都是我一生最美好、最深刻的回忆。

此刻,内心泛起涟漪,想表达的感谢太多太多。

首先,感谢科大,让我有幸能在中国的顶级学府里度过研究生的时光。在科大,让我感受到了老师的负责与专注,感受到了同学的拼搏和刻苦,更感受到了科大低调和魅力,我为自己是一名科大人而自豪。祝美丽的科大60岁生日快乐!

其次,感谢家人,感谢父母和妹妹的陪伴和鼓励,你们是我最大的精神支柱和前进动力。希望父母可以永远健康顺心,希望妹妹可以快乐成长,无论何时何地,家都是最温暖的港湾,我永远爱你们。

感谢王行甫老师,感谢四年前的相知让我有幸成为您的学生。感谢您一直以来在学术研究上对我的帮助和指导,在生活上对我的包容和鼓励,特别是在毕业论文写作期间,感谢您一遍又一遍不厌其烦地和我讨论,帮我理清写作的思路,完善论文的结构,您的认真负责对我产生了很大影响,希望您永远健康。

感谢513实验室的每一位同学,Ammar师兄,王琳师姐,陈静师姐,卞东海,王磊,陈小奇,曾进,刘苹以及已经毕业的吴立涛师兄。感谢你们的存在,让枯燥的科研不再无趣,让寂静的实验室也充满温馨。在这里,要特别感谢Ammar师兄,和你相处的三年里,我们无话不说,你的言行和品质都深深感染着我,谢谢你对我的照顾和信任,对我的鼓励和支持,未来的日子里,希望你在中国生活顺利,我们永远都是最好的兄弟。特别要感谢陈静师姐,谢谢你在我精神压力最大的时候安慰我、鼓励我,希望你在以后的生活里可以一帆风顺,永远幸福。

感谢335宿舍的每一位舍友:常欢,王志宏,田星,谢谢你们在生活上的照顾和包容。研究生能遇到你们三个是我的荣幸,你们真的很优秀,从你们身上我学到了很多,王志宏的开朗豁达、常欢的淡然自如以及田星的执着,真心希望我们都可以实现自己理想,未来前进的路上我们一起加油。

感谢11系的所有尊敬的老师们,感谢你们一直以来无私地教诲和辛勤地付出,让我们收获了宝贵的知识,您们辛苦了!特别要感谢在座的每一位评委,谢谢您们给予我论文的指导和建议,希望你们永远健康、幸福!

最后的最后,感谢自己,祝福自己,希望自己可以越来越好。

\rightline{李鹏程} 

\rightline{2018年4月于合肥} 

\cleardoublepage
\end{acknowledgements}
